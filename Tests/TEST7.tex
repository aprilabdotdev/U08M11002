\documentclass[12pt, a4paper, oneside]{ctexart}
\usepackage{amsmath, amsthm, amssymb, bm, color, framed, graphicx, hyperref, mathrsfs, float, caption,subfigure}
\usepackage[justification=centering]{caption}

% multi-column
\usepackage{tasks}
% itemize
\NewTasksEnvironment[label=(\arabic*), label-width=3ex]{exercise}
%\settasks{
% label = \theexercise.\arabic* ,
% item-indent = 2em ,
% label-width = 2em ,
% label-offset = 0pt
%}

\everymath{\displaystyle}

\title{\textbf{第七次随堂测试}}
\author{U08M11002 Fall 2023}
\date{2023 年 12 月 15 日}
\linespread{1}
\definecolor{shadecolor}{RGB}{241, 241, 255}

\newcounter{problemname}
\newenvironment{problem}{\stepcounter{problemname}\par\noindent\textbf{题目\arabic{problemname}. }}{\\\par}
\newenvironment{warning}{\begin{shaded}\par\noindent\textbf{提交作业方式:}}{\end{shaded}\par}

\begin{document}

\maketitle

\hspace{1em}

\pagestyle{plain}
  
\begin{problem}
    已知某LTI系统得微分方程  $y'(t)+y(t)=f(t)$ \par
    \begin{exercise}(1)
        \task 若全响应为 $y(t)=[3e^{-t}+2e^{-3t}]u(t),y(0^-)=3$,求零输入响应和零状态响应。
        \task 若 $y(0^-)=10$,求零输入响应。
        \task 若全响应为 $y(t)=[3e^{-t}+2e^{-3t}]u(t),y(0^-)=3$,求 $y'(t)+y(t)=f(t-2)$ 的零状态响应。
        \task 若全响应为 $y(t)=[3e^{-t}+2e^{-3t}]u(t),y(0^-)=3$,求 $y'(t)+y(t)=f'(t)+3f(t)$ 的零状态响应。
    \end{exercise}    
\quad
\end{problem}
	
\end{document}