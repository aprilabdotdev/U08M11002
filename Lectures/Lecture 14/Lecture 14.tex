\documentclass[UTF8]{ctexart}
\usepackage{ctex}
\usepackage{geometry}
\usepackage{enumitem}
\usepackage{indentfirst}
\usepackage{color}
\usepackage{fancyhdr}
\usepackage{amsmath}
\usepackage{graphicx}
\usepackage{amssymb}
\usepackage{tikz}
\usepackage{cases}
\usepackage{array}
\usepackage{mathrsfs}
\usepackage{pgfplots}
\usepackage{tkz-euclide}
\usepackage{ragged2e}
\pgfplotsset{compat=1.17}
\usetikzlibrary{arrows.meta, decorations.pathreplacing, calligraphy}

% 设置纸张和页边距——A4
\geometry{papersize={21cm,29.7cm}}
\geometry{left=3.18cm,right=3.18cm,top=2.54cm,bottom=2.54cm}

% 一级标题靠左
\CTEXsetup[format={\Large\bfseries}]{section}

% 去除页眉
\pagestyle{plain}

% 开始文档内容
\begin{document}

\title{信号与系统课程笔记:Lecture 14}
\author{授课教师:秦雨潇 \\
        笔记记录:李梦薇}
\date{2023 年 11 月 03 日(第九周,周五)}
\maketitle

\section{复习}
\begin{enumerate}[label=(\arabic*),itemindent=0pt,labelindent=\parindent,labelwidth=2em,labelsep=5pt,leftmargin=*]
      \item $f(t)=f(t)*\delta(t)$ \par
            $f(t)=\int_{\mathbb{R}}F(\omega)e^{-i\omega{t}}{\rm{d}}t$
      \item $\delta(t)*h(t)=h(t)$ \par
            $1\cdot{H(\omega)}=H(\omega)\quad$?
      \item $f(t)*h(t)=y(t)$ \par
            $F(\omega)\cdot{H(\omega)}=Y(\omega)$
\end{enumerate}\par

\section{能量谱和功率谱(Parseval's定理)}
\begin{enumerate}[label=(\arabic*),itemindent=0pt,labelindent=\parindent,labelwidth=2em,labelsep=5pt,leftmargin=*]
      \item 能量谱:$G(\omega)=|F(\omega)|^2$ \par
            注意:复数是共轭!
      \item 功率谱:$D(\omega)=\lim_{T\to+\infty}\frac{G(\omega)}{T}$ \par
\end{enumerate}\par

\section{周期信号的傅里叶变换}
\subsection{傅里叶级数(FS)}
$f_T(t)=\sum_{n\to-\infty}^{+\infty}F[n]e^{jn\Omega{t}}$ \par
\subsection{傅里叶变换(FT)}
$f_T(t)=\frac{1}{2\pi}\int_{\mathbb{R}}F(\omega)e^{jn\omega}{\rm{d}}\omega$ \par
\subsection{推导过程}
\noindent
\begin{flalign*}\hspace{2em}
      f_T(t)&=\sum_{n\to-\infty}^{+\infty}F[n]e^{jn\Omega{t}} &\\
\end{flalign*} \par
\noindent
\begin{flalign*}\hspace{2em}
      F(\omega)&=\mathscr{F}\{f_T(t)\} &\\
      &=\mathscr{F}\{\sum_{n\to-\infty}^{+\infty}F[n]e^{jn\Omega{t}}\} &\\
      &=\sum_{n\to-\infty}^{+\infty}F[n]\mathscr{F}\{e^{jn\Omega{t}}\} &\\
      &=2\pi\sum_{n\to-\infty}^{+\infty}F[n]\delta(\omega-n\Omega)
\end{flalign*} \par
\noindent
\begin{flalign*}\hspace{2em}
      2\pi\sum_{n\to-\infty}^{+\infty}F[n]\delta(\omega-n\Omega)=&2\pi{F[0]}\delta(\omega-0\Omega)+ &\\
      &2\pi{F[1]}\delta(\omega-1\Omega)+ &\\
      &2\pi{F[2]}\delta(\omega-2\Omega)+\cdots &\\
\end{flalign*} \par
频谱($F_n/F[n]\rightarrow{F(\omega)}$):
\begin{center}
      \begin{tikzpicture}
        \begin{axis}[
            width=10cm, height=6cm,
            axis lines=center,
            xlabel={$\omega$},
            x label style={at={(ticklabel* cs:1)}, anchor=west},
            xmin=-3.5, xmax=3.5,
            ymin=0, ymax=2,
            xtick={-3, -2, -1, 0, 1, 2, 3},
            ytick=\empty,
            y axis line style={draw=none},
            legend style={at={(1.1,1)}, anchor=north west},
        ]
        \addplot[quiver={u=0,v=0.1}, -{Stealth[length=2.5mm]}] coordinates {
            (-3, 0.9)
            (-2, 1.1)
            (-1, 1.3)
            (0, 1.5)
            (1, 1.3)
            (2, 1.1)
            (3, 0.9)
        };
        \draw (axis cs:-3, 0.9) -- (axis cs:-3, -0.1);
        \draw (axis cs:-2, 1.1) -- (axis cs:-2, -0.1);
        \draw (axis cs:-1, 1.3) -- (axis cs:-1, -0.1);
        \draw (axis cs:0, 1.5) -- (axis cs:0, -0.1);
        \draw (axis cs:1, 1.3) -- (axis cs:1, -0.1);
        \draw (axis cs:2, 1.1) -- (axis cs:2, -0.1);
        \draw (axis cs:3, 0.9) -- (axis cs:3, -0.1);
        \node[anchor=south] at (axis cs:0,1.5) {$2\pi{F[0]}$};
        \node[anchor=south] at (axis cs:1,1.3) {$2\pi{F[1]}$};
        \node[anchor=south] at (axis cs:2,1.1) {$2\pi{F[2]}$};
        \node[anchor=south] at (axis cs:3,0.9) {$2\pi{F[3]}$};
        \end{axis}
      \end{tikzpicture}
\end{center} \par

\section{常见信号的FT}
\begin{enumerate}[label=(\arabic*),itemindent=0pt,labelindent=\parindent,labelwidth=2em,labelsep=5pt,leftmargin=*]
      \item $1\rightleftharpoons2\pi\delta(\omega)$
      \item $e^{j\omega_0t}\rightleftharpoons2\pi\delta(\omega-\omega_0)$
      \item $\cos(\omega_0t)\rightleftharpoons\pi\delta(\omega-\omega_0)+\pi\delta(\omega+\omega_0)$
\end{enumerate}\par

\section{扩展}
\begin{tabular}{ c c c c }
      $\quad$周期信号$\quad$ & $\rightleftharpoons$ & $\quad$离散信号$\quad$ & $\quad$ \\
      $T$(周期) & $\times$ & $\Delta{S}$(采样频率) & $=2\pi$(常数)\\
\end{tabular} \par
(CTFT:连续非周期;DTFT:离散非周期;DFT:离散周期;CTFS:时域连续周期)\par
\begin{enumerate}[label=(\arabic*),itemindent=0pt,labelindent=\parindent,labelwidth=2em,labelsep=5pt,leftmargin=*]
      \item 周期性的$\delta(t)$,周期为$T$。$\quad\rightarrow\quad$“离散”
            \begin{flalign*}
                  &2\pi\sum_{n\to-\infty}^{+\infty}\frac{1}{T}\delta(\omega-n\Omega) \qquad \Omega=\frac{2\pi}{T}\text{,}T\text{已知}&\\
                  &=\frac{1}{\Omega}\sum_{n\to-\infty}^{+\infty}\delta(\omega-n\Omega) &\\
                  &=\frac{1}{\Omega}\delta_\Omega(\omega) &\\
            \end{flalign*}
            \begin{center}
                  \begin{tikzpicture}
                    \begin{axis}[
                        width=8cm, height=4cm,
                        axis lines=center,
                        xlabel={$\omega$},
                        x label style={at={(ticklabel* cs:1)}, anchor=west},
                        xmin=-2.5, xmax=1.5,
                        ymin=0, ymax=2,
                        xtick={-2, -1, 0, 1},
                        xticklabels={ , , , },
                        ytick=\empty,
                        y axis line style={draw=none},
                        legend style={at={(1.1,1)}, anchor=north west},
                    ]
                    \addplot[quiver={u=0,v=0.1}, -{Stealth[length=2.5mm]}] coordinates {
                        (-2, 1.5)
                        (-1, 1.5)
                        (0, 1.5)
                        (1, 1.5)
                    };
                    \draw (axis cs:-2, 1.5) -- (axis cs:-2, -0.1);
                    \draw (axis cs:-1, 1.5) -- (axis cs:-1, -0.1);
                    \draw (axis cs:0, 1.5) -- (axis cs:0, -0.1);
                    \draw (axis cs:1, 1.5) -- (axis cs:1, -0.1);
                    \node[anchor=east] at (axis cs:-2,1.5) {$\frac{1}{\Omega}$};
                    \draw[decorate, decoration={calligraphic brace, amplitude=5pt}] (axis cs:-1, 0.05) -- (axis cs:0, 0.05) node[midway, above=5pt] {$\Omega$};
                    \end{axis}
                  \end{tikzpicture}
            \end{center}
      \item 可以理解为直流信号的采样,例如:离散信号。$\quad\rightarrow\quad$“周期” \par
            $1\rightleftharpoons2\pi\delta(\omega)$ \par
            $\frac{1}{T}\sum_{n\to-\infty}^{+\infty}2\pi\delta(\omega-n\Omega)=\frac{1}{\Omega}\sum_{n\to-\infty}^{+\infty}\delta(\omega-n\Omega)$
      \item $\cos(\frac{2\pi}{T}t)$的采样,离散信号。$\quad\rightarrow\quad$“周期” 
\end{enumerate}\par

\end{document}