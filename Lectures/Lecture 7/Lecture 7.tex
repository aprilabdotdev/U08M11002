\documentclass[UTF8]{ctexart}
\usepackage{enumitem}
%插入数学公式
\usepackage{amsmath}
\usepackage{amssymb}
\usepackage{graphicx} % 图片插入
%设置纸张和页边距 A4
\usepackage{geometry}
\geometry{papersize={21cm,29.7cm}}
\geometry{left=3.18cm,right=3.18cm,top=2.54cm,bottom=2.54cm}
% 一级标题靠左
\CTEXsetup[format={\Large\bfseries}]{section}
% 去除页眉
\pagestyle{plain}
% %设置行间距 1.5倍行距
% \usepackage{setspace}
% \onehalfspacing
%设置段间距
\addtolength{\parskip}{.4em}
\title{信号与系统课程笔记:Lecture 7}
\author{授课教师:秦雨潇 \\
笔记记录:曹时成}
\date{2023年10月13日(第六周,周五)}

\begin{document}
\maketitle
\section{课堂回顾:如何判断是不是LTI?}
例:$f(t)=e^{-t}\int_{-\infty}^{t} f(\tau ) e^\tau \,d\tau  $ \par
即判断:$f(t-t_0)=y(t-t_0) $是否成立?成立则为LTI,不成立则不是\par
在这里:\par
(1)$f(t-t_0)=e^{-t}\int_{-\infty}^{t}f(\tau -t_0)e^\tau\,d\tau   $ \par
(2)$y(t-t_0)=e^{-(t-t_0)}\int_{-\infty}^{t-t_0}f(\tau )e^\tau\,d\tau   $ \par
进行积分换元化简,可推导(1)与(2)是相等的,该例子为LTI。\par

\section{矢量的分解}
我们希望,对于我们的矢量$\vec{A}\in \mathbb{R} ^{n\times 1}$,可以分解为“子矢量”$\{\vec{v}_1,\vec{v}_2,\ldots ,\vec{v}_n \} $,且“子矢量”满足如下性质,则这组“子矢量”被称为一组“正交基”\par
(1)$\vec{v}_\perp \vec{v}_j$ \qquad $i\neq j$ ,\quad $i,j\leqq n$\par
(2)$\parallel \vec{v}_i \parallel = 1 $ $\leftrightarrow$ 类似于 $("\delta (t),\int_\mathbb{R} \delta (t)\,dt=1")$\par
(3)对于 $\vec{A}\in \mathbb{R} ^{n\times 1}$ ,$span\{\vec{v}_1,\vec{v}_2,\ldots ,\vec{v}_n \}  \in \mathbb{R} ^{n\times 1} $ \par
则必有:$vec{A}=C_1 \vec{v}_1+C_2 \vec{v}_2+,\ldots ,C_n \vec{v}_n$ \par
其中:$C_i=\frac{\langle \vec{A},\vec{v}_i \rangle }{\|\vec{v}_i \Vert^2 }  $\par
 \qquad $C_i$是$\vec{A}$ 在$\vec{v}_i$上的投影 \par

 \section{信号的分解}
 我们希望,对于我们的一个信号$f(t)$,可以分解为“子信号”$\{\vec{v}_1,\vec{v}_2,\ldots ,\vec{v}_n \} $,且“子信号”满足如下性质,则这组“子信号”被称为一组“正交完备集”\par
 (1)“子信号”是正交的 \par
 (2)“子信号”是单位的 \par
 \[  \int_{\tau _a}^{\tau _b} \vec{v }_i \vec{v }_j  \,d\tau  =\left\{ \begin{array}{rcl}
    0 & & {i\neq j,  \qquad  “ortho”}  \\
    1 & & {i= j,    \qquad    “normal”} \\
    \end{array} \right. \]
(3)完备性 \par
如果在$\{\vec{v}_1,\vec{v}_2,\ldots ,\vec{v}_n \} $以外,不存在任何$g(t)\neq 0$,满足$\langle \vec{v }_i (t), g (t)\rangle =0,(t\in [t_a,t_b ] )$,则称,$\{\vec{v}_1,\vec{v}_2,\ldots ,\vec{v}_n \} $为$f(t)$的正交完备集。\par
维度n有可能是无限的。 \par
则有:$f(t)=C_1 \vec{v}_1(t)+C_2 \vec{v}_2(t)+,\ldots ,C_n \vec{v}_n(t)$ \par
其中:$C_i=\frac{\langle f(t),\vec{v}_i(t) \rangle }{\|\vec{v}_i \Vert^2 }  $\par
\section{思考题:下面这四个特殊信号是不是正交完备集}
(1)$\delta $函数:$\delta [k-\tau ]$ \par
(2)阶跃函数:$u [k-\tau ]$ \par
(3)门函数:$rect [k-\tau ]$ \par
(4)随机分布函数:$N(\mu ,\delta ): f(x)=\frac{1}{\sigma \sqrt{2\pi }} e^{-\frac{(x-\mu )^2}{2\sigma ^2} } $ \par

\end{document}