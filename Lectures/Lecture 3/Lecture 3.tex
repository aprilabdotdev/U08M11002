\documentclass[UTF8]{ctexart}
\usepackage{enumitem}
%插入数学公式
\usepackage{amsmath}
\usepackage{amssymb}
% \usepackage{graphicx} %\includegraphics{a.jpg} 图片插入
%设置纸张和页边距 A4
\usepackage{geometry}
\geometry{papersize={21cm,29.7cm}}
\geometry{left=3.18cm,right=3.18cm,top=2.54cm,bottom=2.54cm}
% 一级标题靠左
\CTEXsetup[format={\Large\bfseries}]{section}
% 去除页眉
\pagestyle{plain}
% %设置行间距 1.5倍行距
% \usepackage{setspace}
% \onehalfspacing
%设置段间距
\addtolength{\parskip}{.4em}
\title{信号与系统课程笔记:Lecture 3}
\author{授课教师:秦雨潇 \\
笔记记录:曹时成}
\date{2023年9月20日(第三周,周三)}

\begin{document}
\maketitle

\section{课堂复习}
\subsection{线性时不变系统(Linear Time Invariant system)}
\subparagraph{if 有:} \par
$f_1(t)\to h(t)\to y_1(t)$;$f_2(t)\to h(t)\to y_2(t)$ \par
\subparagraph{则有:}
\textbf{齐次性:}
$a\cdot f(t)\to a\cdot y(t)$\par
\textbf{\qquad \; \; 叠加性:}
$f_1(t)+f_2(t)=y_1(t)+y_2(t)$

\subsection{例题:is LTI or not?}
\subparagraph{例1:} 
$y(t)=3x(t)+4$\par
\textbf{解:}
$3[a\ x(t)] +4\neq a[ 3x(t)+4]  $\par
\subparagraph{例2:} 
$y(t)=4[ x(t)] ^2$\par
\textbf{解:}
$ay(t)=4a[x(t) ]^2\neq 4[ ax(t)] ^2   $
\subparagraph{此外:微分,积分,延时器都是LTI \\ }
\textbf{ \;  if 有:} 
$f(t)\to h(t)\to y(t)$\par
\textbf{则有:} 
$f^{'} (t)\to y^{'}(t)$;$\int_{-\infty }^{\tau  } f(t) \,dt \to \int_{-\infty }^{\tau } y(t) \,dt$ \par
\subparagraph{作业:证明微分,积分,延时器都是LTI}

\section{信号的分类}
\begin{enumerate}[itemindent=2em,label=(\arabic*)]
    \item 确定信号(deterministic)和随机信号(stochastic)
    \item 连续信号(continuous)和离散信号(discrete)
    \item 周期信号(periodic)和非周期信号(non-periodic)
    \[ f(t)=f(t+nt),n\in Z. \] \par
    \qquad 周期函数的叠加不一定是周期函数
    \item 功率信号(power)和能量信号(energy)
    \[  E\triangleq \int_{-\infty}^{\infty} | f(t)\vert ^2 \,dt ; \]
    \[  P\triangleq \lim_{T \to \infty}  \int_{-T}^{T} | f(t)\vert ^2 \,dt .\]
    \item 因果信号(causal $\& $ result)
    \item 实数信号(real)和复数信号(complex)
\end{enumerate}

\end{document}