\documentclass[UTF8]{ctexart}
\usepackage{ctex}
\usepackage{geometry}
\usepackage{enumitem}
\usepackage{indentfirst}
\usepackage{color}
\usepackage{fancyhdr}
\usepackage{amsmath}
\usepackage{graphicx}
\usepackage{amssymb}
\usepackage{tikz}
\usepackage{cases}
\usepackage{array}
\usepackage{mathrsfs}
\usepackage{extarrows}

% 设置纸张和页边距——A4
\geometry{papersize={21cm,29.7cm}}
\geometry{left=3.18cm,right=3.18cm,top=2.54cm,bottom=2.54cm}

% 一级标题靠左
\CTEXsetup[format={\Large\bfseries}]{section}

% 去除页眉
\pagestyle{plain}

% 开始文档内容
\begin{document}

\title{信号与系统课程笔记:Lecture 19-20}
\author{授课教师:秦雨潇 \\
        笔记记录:李梦薇}
\date{2023 年 11 月 17 日(第十一周,周五)}
\maketitle

\section{复习}
\begin{enumerate}[label=(\arabic*),itemindent=0pt,labelindent=\parindent,labelwidth=2em,labelsep=5pt,leftmargin=*]
      \item LT与FT的本质相同$\qquad{f(t)e^{-\sigma{t}}U(t)}$
      \item LT是FT的拓展$\qquad{\omega\rightarrow{s=\sigma+j\omega}}$
      \item $F(s)=\int_{0}^{+\infty}f(t)e^{-st}{\rm{d}}t$,$t\geqslant0$ \par
            $f(t)=\int_{\sigma-j\infty}^{\sigma+j\infty}F(s)e^{st}{\rm{d}}s$,$t\geqslant0$
\end{enumerate}\par

\section{拉普拉斯变换(Laplace Transfrom,LT)的性质}
\begin{enumerate}[label=(\arabic*),itemindent=0pt,labelindent=\parindent,labelwidth=2em,labelsep=5pt,leftmargin=*]
      \item (相同)线性:if $f_1(t)\rightleftharpoons{F_1(s)}$,$f_2(t)\rightleftharpoons{F_2(s)}$ \par
            \begin{itemize}[label=,left=6em]
              \item than $Af_1(t)+Bf_2(t)\rightleftharpoons{AF_1(s)}+BF_2(s)$
              \item $f_1(t):\sigma>\sigma_1$,$f_2(t):\sigma>\sigma_2$,则$\sigma>\max(\sigma_1,\sigma_2)$
            \end{itemize}
      \item (相同)尺度变换:if $f(t)\rightleftharpoons{F(s)}$ \par
            \begin{itemize}[label=,left=8em]
              \item than $f(\alpha{t})\rightleftharpoons\frac{1}{\alpha}{F(\frac{s}{\alpha})}$,$\sigma>\alpha\sigma_0$
            \end{itemize}
      \item (相似)时移:$f(t-{t_0})\rightleftharpoons{e^{-st_0}F(s)}$,$t_0\geqslant0$
      \item (相同)复频移:$f(t)e^{st_0}\rightleftharpoons{F(s-s_0)}$,$\sigma\geqslant\sigma_0$(原来的)$+\sigma_0$(对应$s_0$)
      \item $\bigstar$\textbf{(相似)时域微分:}$f'(t)\rightleftharpoons{sF(s)-f(0^-)}$ \par
            证明:$\int_{0^-}^{+\infty}\frac{{\rm{d}}}{{\rm{d}}t}f(t)\cdot{e^{-st}}{\rm{d}}t$
            \begin{itemize}[label=,left=2.5em]
              \item $=\int_{0^-}^{+\infty}\{\frac{{\rm{d}}}{{\rm{d}}t}[f(t)\cdot{e^{-st}}]+sf(t)e^{-st}\}{\rm{d}}t$
              \item $=\int_{0^-}^{+\infty}\frac{{\rm{d}}}{{\rm{d}}t}[f(t)\cdot{e^{-st}}]{\rm{d}}t+s\int_{0^-}^{+\infty}f(t)e^{-st}{\rm{d}}t$
              \item $=f(t)e^{-st}\big{|}_{0^-}^{+\infty}+sF(s)$
              \item $=0-f(0^-)+sF(s)$
              \item $=sF(s)-f(0^-)$
            \end{itemize} \par
            扩展:求$f''(t)$
            \begin{itemize}[label=,left=2.5em]
              \item 令$g(t)=f'(t)$,则求$g'(t)$的LT
              \item $g'(t)=sG(s)-f'(0^-)$,其中$G(s)=\mathscr{L}\{f'(t)\}=sF(s)-f(0^-)$
              \item 因此,$f''(t)=s^2F(s)-sf(0^-)-f'(0^-)$
            \end{itemize} \par
            总结:$f^n(t)=s^nF(s)-s^{n-1}f(0^-)-s^{n-2}f'(0^-)-\cdots-sf^{n-2}(0^-)-f^{n-1}(0^-)$
      \item 时域积分:$\int_{0^-}^{t}f(\tau){\rm{d}}\tau\rightleftharpoons\frac{1}{s}F(s)$ \par
            \begin{itemize}[label=,left=4.5em]
              \item $\int_{-\infty}^{t}f(\tau){\rm{d}}\tau\rightleftharpoons\frac{1}{s}F(s)+\frac{1}{s}f^{-1}(0^-)$
            \end{itemize}
      \item s域微分:$(-t)^nf(t)\rightleftharpoons\frac{{\rm{d}}^nF(s)}{{\rm{d}}s^n}$
      \item s域积分:$\frac{f(t)}{t}\rightleftharpoons\int_{s}^{+\infty}F(\lambda){\rm{d}}\lambda$
      \item (相同)卷积:$f_1(t)*f_2(t)\rightleftharpoons{F_1(s)F_2(t)}$,$\sigma>{\rm{max}}(\sigma_1,\sigma_2)$ \par
            注意:$[f_1(t)U(t)]*[f_2(t)U(t)]\quad$即$f_1(t),f_2(t)$为因果信号。
      \item 对称性:没有这个性质!
      \item 初值,终值定理:$\lim_{s\rightarrow0}sF(s)=f(+\infty)$,$\lim_{s\rightarrow+\infty}sF(s)=f(0^+)$
            \noindent
            \begin{flalign*}\hspace{0em}
              \text{证明:}f'(t)&=sF(s)-f(0^-) &\\
              &=\int_{0^-}^{+\infty}\frac{{\rm{d}}}{{\rm{d}}t}f(t)e^{-st}{\rm{d}}t &\\
              &=\int_{0^-}^{0^+}\frac{{\rm{d}}}{{\rm{d}}t}f(t)e^{-st}{\rm{d}}t+\int_{0^+}^{+\infty}\frac{{\rm{d}}}{{\rm{d}}t}f(t)e^{-st}{\rm{d}}t &\\
              &\xlongequal{s\rightarrow+\infty}f(0^+)-f(0^-)+0-0 &\\
              &=f(0^+)-f(0^-)
            \end{flalign*} \par
            \begin{itemize}[label=,left=2.5em]
              \item 因此$sF(s)-f(0^-)=f(0^+)-f(0^-)$
              \item 则$sF(s)=f(0^+)$,$(s\rightarrow+\infty)$
            \end{itemize}
      \item Bonus (1):$\int_{-\infty}^{+\infty}f(t){\rm{d}}t=F(0)$
      \item Bonus (2):$\int_{\mathbb{R}}{\rm{sinc}}(x){\rm{d}}x=1$,$\int_{\mathbb{R}}|{\rm{sinc}}(x)|{\rm{d}}x=+\infty$
\end{enumerate}\par

\section{例题}
\begin{enumerate}[label=(\arabic*),itemindent=0pt,labelindent=\parindent,labelwidth=2em,labelsep=5pt,leftmargin=*]
      \item $y''(t)+5y'(t)+4y(t)=0$,$y(0^-)=2$,$y'(0^-)=-5$,求解系统。 \par
            第一步:LT \par
            \noindent
            \begin{flalign*}
              s^2Y(s)-sy(0^-)-y'(0^-)+5[sY(s)-y(0^-)]+4Y(s)&=0 &\\
              (s^2+5s+4)Y(s)-(s+5)y(0^-)-y'(0^-)&=0 &\\
              (s^2+5s+4)Y(s)-(2s+5)&=0 &\\
              Y(s)&=\frac{2s+5}{s^2+5s+4}
            \end{flalign*}
            第二步:Inverse LT \par
            \noindent
            \begin{flalign*}
              y(t)&=\mathscr{L}^{-1}\{Y(s)\} &\\
              &=\mathscr{L}^{-1}\{\frac{2s+5}{s^2+5s+4}\} &\\
              &=\mathscr{L}^{-1}\{k_1\frac{1}{s+1}+k_2\frac{1}{s+4}\} &\\
              &=k_1e^{-t}+k_2e^{-4t}
            \end{flalign*}
      \item $y''(t)+5y'(t)+4y(t)=U(t)$,$y(0^-)=2$,$y'(0^-)=-5$,求解系统。 \par
            第一步:LT \par
            \noindent
            \begin{flalign*}
              s^2Y(s)-sy(0^-)-y'(0^-)+5[sY(s)-y(0^-)]+4Y(s)&=\frac{1}{s} &\\
              (s^2+5s+4)Y(s)-(2s+5)&=\frac{1}{s} &\\
              (s^3+5s^2+4s)Y(s)-(2s^2+5s)&=1 &\\
              Y(s)&=\frac{2s^2+5s+1}{s^3+5s^2+4s}
            \end{flalign*}
            第二步:Inverse LT \par
            \noindent
            \begin{flalign*}
              y(t)&=\mathscr{L}^{-1}\{Y(s)\} &\\
              &=\mathscr{L}^{-1}\{\frac{2s^2+5s+1}{s^3+5s^2+4s}\} &\\
              &=\mathscr{L}^{-1}\{\sum_{i=1}^{3}\frac{q_i}{s+p_i}\} &\\
              &=\sum_{i=1}^{3}q_ie^{-p_it}
            \end{flalign*}
      \item $y''(t)+5y'(t)+6y(t)=2f'(t)+6f(t)$,$y(0^-)=1$,$y'(0^-)=-1$,$f(t)=5\cos(t)U(t)$,求完全响应。 \par
            第一步:LT \par
            \noindent
            \begin{flalign*}
              Y(s)&=\frac{s^2y(0^-)+y'(0^-)+sy(0^-)}{s^2+5s+6}+\frac{2(s+3)}{s^2+5s+6}F(s) &\\
              &=\frac{k_1}{s+2}+\frac{k_2}{s+3}+\frac{k_3}{s+2}+\frac{k_4}{s+1}
            \end{flalign*}
            其中:第一项$\frac{s^2y(0^-)+y'(0^-)+sy(0^-)}{s^2+5s+6}$为“零输入响应”项;\par
            \qquad\quad 第二项$\frac{2(s+3)}{s^2+5s+6}F(s)$为“零状态响应”项;\par
            \qquad\quad 第一项和第二项分母$s^2+5s+6$为“系统”;\par
            \qquad\quad 第一项分子$s^2y(0^-)+y'(0^-)+sy(0^-)$为“初始条件”;\par
            \qquad\quad 第二项分子$2(s+3)F(s)$为“激励”。\par
            第二步:Inverse LT \par
            $\mathscr{L}^{-1}\{Y(s)\}=[k_1e^{-2t}+k_2e^{-3t}+k_3e^{-2t}+k_4\cos(t-\varphi_4)]U(t) $\par
            其中:前两项$k_1e^{-2t}+k_2e^{-3t}$为“零输入”;\par
            \qquad\quad 后两项$k_3e^{-2t}+k_4\cos(t-\varphi_4)$为“零状态”。\par
            或者:前三项$k_1e^{-2t}+k_2e^{-3t}+k_3e^{-2t}$为“暂态分量”;\par
            \qquad\quad 最后一项$k_4\cos(t-\varphi_4)$为“稳态分量”。\par
\end{enumerate}\par

\end{document}